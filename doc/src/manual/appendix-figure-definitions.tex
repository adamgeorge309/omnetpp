\appendixchapter{Figure Definitions}
\label{cha:figure-definitions}

This appendix provides a reference for defining figures in NED files.

\section{Built-in Figure Types}
\label{sec:figure-definitions:figure-types}

The following table lists the figure types supported by {\opp}.

\begin{longtable}{|l|l|}
\hline
\tabheadcol
\tbf{@figure type} & \tbf{C++ class}            \\\hline
\tbf{line}         & \cclass{cLineFigure}       \\\hline
\tbf{arc}          & \cclass{cArcFigure}        \\\hline
\tbf{polyline}     & \cclass{cPolylineFigure}   \\\hline
\tbf{rectangle}    & \cclass{cRectangleFigure}  \\\hline
\tbf{oval}         & \cclass{cOvalFigure}       \\\hline
\tbf{ring}         & \cclass{cRingFigure}       \\\hline
\tbf{pieslice}     & \cclass{cPieSliceFigure}   \\\hline
\tbf{polygon}      & \cclass{cPolygonFigure}    \\\hline
\tbf{path}         & \cclass{cPathFigure}       \\\hline
\tbf{text}         & \cclass{cTextFigure}       \\\hline
\tbf{label}        & \cclass{cLabelFigure}      \\\hline
\tbf{image}        & \cclass{cImageFigure}      \\\hline
\tbf{icon}         & \cclass{cIconFigure}       \\\hline
\tbf{pixmap}       & \cclass{cPixmapFigure}     \\\hline
\tbf{group}        & \cclass{cGroupFigure}      \\\hline
\end{longtable}

Additional figure types can be defined with the
\ttt{custom:\textit{<type>}} syntax; see the \textit{FigureType} below.

\section{Attribute Types}
\label{sec:figure-definitions:attribute-types}

This section lists the available attribute types and their value syntaxes.

\begin{description}

\item[bool]: \\
\ttt{true} or \ttt{false}.

\item[int]: \\
An integer.

\item[double]: \\
A real number.

\item[double01]: \\
A real number in the interval [0,1].

\item[degrees]: \\
A real number that will be interpreted as degrees.

\item[string]: \\
A string. It only needs to be enclosed in quotes if it contains a comma,
a semicolon, an unmatched close parenthesis, or any other character that prevents
it from being properly parsed as a property value.

\item[Anchor]: \\
\ttt{c}, \ttt{center}, \ttt{n}, \ttt{e}, \ttt{s}, \ttt{w}, \ttt{nw},
\ttt{ne}, \ttt{se}, \ttt{sw}, \ttt{start}, \ttt{middle}, or \ttt{end}.
The last three are only valid for text figures.

\item[Arrowhead]: \\
\ttt{none}, \ttt{simple}, \ttt{triangle}, or \ttt{barbed}.

\item[CapStyle]: \\
\ttt{butt}, \ttt{square}, or \ttt{round}.

\item[Color]: \\
A color in HTML format (\textit{\#rrggbb}), a color in
HSB format (\textit{@hhssbb}), or a valid SVG color name.

\item[Dimensions]: \textit{width, height} \\
Size given as width and height.

\item[FigureType]: \\
One of the built-in figure types (e.g. \ttt{line} or \ttt{arc}, see
\ref{sec:figure-definitions:figure-types}), or a figure type registered
with \fmac{Register\_Figure()}.

\item[FillRule]: \\
\ttt{evenodd} or \ttt{nonzero}.

\item[Font]: \textit{typeface, size, style} \\
All three items are optional. \textit{size} is the font size in points.
\textit{style} is a space-separated list of zero or more of the following
words: \ttt{normal}, \ttt{bold}, \ttt{italic}, \ttt{underline}.

\item[ImageName]: \\
The name of an image.

\item[Interpolation]: \\
\ttt{none}, \ttt{fast}, or \ttt{best}.

\item[JoinStyle]: \\
\ttt{bevel}, \ttt{miter}, or \ttt{round}.

\item[LineStyle]: \\
\ttt{solid}, \ttt{dotted}, or \ttt{dashed}.

\item[Point]: \textit{x, y} \\
A point with coordinates $(x,y)$.

\item[Point2]: \textit{x1, y1, x2, y2} \\
Two points: $(x1,y1)$ and $(x2,y2)$.

\item[PointList]: \textit{x1, y1, x2, y2, x3, y3...} \\
A list of points such as $(x1,y1)$, $(x2,y2)$, $(x3,y3)$, etc.

\item[Rectangle]: \textit{x, y, width, height} \\
A rectangle given by its top-left corner and dimensions.

\item[TagList]: \textit{tag1, tag2, tag3...} \\
A list of string tags.

\item[Tint]: \textit{Color, double01} \\
Specifies the tint color and the amount of tinting for images.

\item[Transform]: \\
One or more transform steps. A step can be one of the following: \\
\ttt{translate($x$, $y$)}, \\
\ttt{rotate($deg$)}, \\
\ttt{rotate($deg$, $centerx$, $centery$)}, \\
\ttt{scale($s$)}, \ttt{scale($sx$, $sy$)}, \\
\ttt{scale($s$, $centerx$, $centery$)}, \\
\ttt{scale($sx$, $sy$, $centerx$, $centery$)}, \\
\ttt{skewx($coeff$)}, \\
\ttt{skewx($coeff$, $centery$)}, \\
\ttt{skewy($coeff$)}, \\
\ttt{skewy($coeff$, $centerx$)}, \\
\ttt{matrix($a$, $b$, $c$, $d$, $t1$, $t2$)}

\end{description}



\section{Figure Attributes}
\label{sec:figure-definitions:figure-attributes}

This section lists the attributes accepted by individual figure types.
Types enclosed in parentheses are abstract types that cannot be used
directly; their sole purpose is to provide a base for more specialized types.

\begin{flushleft}
\begin{description}

\item[(figure)]: \\
    \ttt{type}=\textit{<FigureType>};
    \ttt{visible}=\textit{<bool>};
    \ttt{tags}=\textit{<TagList>};
    \ttt{childZ}=\textit{<int>};
    \ttt{transform}=\textit{<Transform>};

\item[(abstractLine)]: figure \\
    \ttt{lineColor}=\textit{<Color>};
    \ttt{lineStyle}=\textit{<LineStyle>};
    \ttt{lineWidth}=\textit{<double>};
    \ttt{lineOpacity}=\textit{<double>};
    \ttt{capStyle}=\textit{<CapStyle>};
    \ttt{startArrowhead}=\textit{<Arrowhead>};
    \ttt{endArrowhead}=\textit{<Arrowhead>};
    \ttt{zoomLineWidth}=\textit{<bool>};

\item[line]: abstractLine \\
    \ttt{points}=\textit{<Point2>}

\item[arc]: abstractLine \\
    \ttt{bounds}=\textit{<Rectangle>}
    \ttt{pos}=\textit{<Point>};
    \ttt{size}=\textit{<Dimensions>};
    \ttt{anchor}=\textit{<Anchor>};
    \ttt{startAngle}=\textit{<degrees>};
    \ttt{endAngle}=\textit{<degrees>}

\item[polyline]: abstractLine \\
    \ttt{points}=\textit{<PointList>};
    \ttt{smooth}=\textit{<bool>};
    \ttt{joinstyle}=\textit{<JoinStyle>}

\item[(abstractShape)]: figure \\
    \ttt{lineColor}=\textit{<Color>};
    \ttt{fillColor}=\textit{<Color>};
    \ttt{lineStyle}=\textit{<LineStyle>};
    \ttt{lineWidth}=\textit{<double>};
    \ttt{lineOpacity}=\textit{<double01>};
    \ttt{fillOpacity}=\textit{<double01>};
    \ttt{zoomLineWidth}=\textit{<bool>}

\item[rectangle]: abstractShape \\
    \ttt{bounds}=\textit{<Rectangle>}
    \ttt{pos}=\textit{<Point>};
    \ttt{size}=\textit{<Dimensions>};
    \ttt{anchor}=\textit{<Anchor>};
    \ttt{cornerRadius}=\textit{<double>|<Dimensions>}

\item[oval]: abstractShape \\
    \ttt{bounds}=\textit{<Rectangle>}
    \ttt{pos}=\textit{<Point>};
    \ttt{size}=\textit{<Dimensions>};
    \ttt{anchor}=\textit{<Anchor>}

\item[ring]: abstractShape \\
    \ttt{bounds}=\textit{<Rectangle>}
    \ttt{pos}=\textit{<Point>};
    \ttt{size}=\textit{<Dimensions>};
    \ttt{anchor}=\textit{<Anchor>};
    \ttt{innerSize}=\textit{<Dimensions>}

\item[pieslice]: abstractShape \\
    \ttt{bounds}=\textit{<Rectangle>}
    \ttt{pos}=\textit{<Point>};
    \ttt{size}=\textit{<Dimensions>};
    \ttt{anchor}=\textit{<Anchor>};
    \ttt{startAngle}=\textit{<degrees>};
    \ttt{endAngle}=\textit{<degrees>}

\item[polygon]: abstractShape \\
    \ttt{points}=\textit{<PointList>};
    \ttt{smooth}=\textit{<bool>};
    \ttt{joinStyle}=\textit{<JoinStyle>};
    \ttt{fillRule}=\textit{<FillRule>}

\item[path]: abstractShape \\
    \ttt{path}=\textit{<string>};
    \ttt{offset}=\textit{<Point>};
    \ttt{joinStyle}=\textit{<JoinStyle>};
    \ttt{capStyle}=\textit{<CapStyle>};
    \ttt{fillRule}=\textit{<FillRule>}

\item[(abstractText)]: figure \\
    \ttt{pos}=\textit{<Point>};
    \ttt{anchor}=\textit{<Anchor>}
    \ttt{text}=\textit{<string>};
    \ttt{font}=\textit{<Font>};
    \ttt{opacity}=\textit{<double01>};
    \ttt{color}=\textit{<Color>};

\item[label]: abstractText \\
    \ttt{angle}=\textit{<degrees>};

\item[text]: abstractText

\item[(abstractImage)]: figure \\
    \ttt{bounds}=\textit{<Rectangle>}
    \ttt{pos}=\textit{<Point>};
    \ttt{size}=\textit{<Dimensions>};
    \ttt{anchor}=\textit{<Anchor>};
    \ttt{interpolation}=\textit{<Interpolation>};
    \ttt{opacity}=\textit{<double01>};
    \ttt{tint}=\textit{<Tint>}

\item[image]: abstractImage \\
    \ttt{image}=\textit{<ImageName>}

\item[icon]: abstractImage \\
    \ttt{image}=\textit{<ImageName>}

\item[pixmap]: abstractImage \\
    \ttt{resolution}=\textit{<Dimensions>}

\end{description}
\end{flushleft}


%%% Local Variables:
%%% mode: latex
%%% TeX-master: "usman"
%%% End:




















