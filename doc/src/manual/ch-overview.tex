\chapter{Overview}
\label{cha:overview}


\section{Modeling Concepts}
\label{sec:overview:modeling-concepts}

An {\opp} model consists of modules that communicate through message passing.
The active modules are called \textit{simple modules}; they are written in C++,
using the simulation class library. Simple modules can be grouped into
\textit{compound modules} and so on; there is no limit to the number of hierarchy levels.
The entire model, referred to as a network in {\opp}, is itself a compound module.
Messages can be sent either via connections that span
modules or directly to other modules.

In Fig. \ref{fig:ch-overview:modules}, boxes represent simple modules
(with gray background) and compound modules.
Arrows connecting the small boxes represent connections and gates.

\begin{figure}[htbp]
  \begin{center}
    \includesvg[scale=0.9]{figures/over-modules}
    \caption{Simple and compound modules}
    \label{fig:ch-overview:modules}
  \end{center}
\end{figure}


Modules communicate with messages that can contain arbitrary
data, in addition to the usual attributes such as a timestamp.
Simple modules typically send messages through gates, but it is also
possible to send them directly to their destination modules. Gates are the
input and output interfaces of modules: messages are sent through
output gates and arrive through input gates. An input gate and output gate
can be linked by a connection. Connections are created within a single
level of module hierarchy; within a compound module, the gates of
two submodules, or a gate of one submodule and a gate of the compound
module can be connected. Connections spanning hierarchy levels are
not permitted, as they would hinder model reuse. Because of the hierarchical
structure of the model, messages typically travel through a chain of
connections, starting and arriving in simple modules. Compound modules act like
"cardboard boxes" in the model, transparently relaying messages between
their inner realm and the outside world. Parameters such as propagation delay,
data rate, and bit error rate can be assigned to connections. One can also
define connection types with specific properties (referred to as channels) and
reuse them in several places. Modules can have parameters. Parameters are
used mainly to pass configuration data to simple modules, and to help
define the model's topology. Parameters can hold string, numeric, or boolean
values. Because parameters are represented as objects in the program,
parameters -- in addition to holding constants -- may also act as
sources of random numbers, with the actual distributions provided by the
model configuration. They may interactively prompt the user for a value,
and they may also hold expressions referencing other parameters. Compound
modules may pass parameters or expressions of parameters to their
submodules.


{\opp} provides efficient tools for the user to describe the
structure of the actual system. Some of the main features are as follows:

\begin{itemize}
\item hierarchically nested modules
\item modules are instances of module types
\item modules communicate with messages through channels
\item flexible module parameters
\item topology description language
\end{itemize}

\subsection{Hierarchical Modules}
\label{sec:overview:hierarchical-modules}


An {\opp} model consists of hierarchically nested
modules\index{module!hierarchy} that communicate by passing
messages to each other.
{\opp} models are often referred to as \textit{networks}. The top
level module is the \textit{system module}.  The system module
contains \textit{submodules} that can also contain submodules
themselves (Fig. \ref{fig:ch-overview:modules}). The depth of module
nesting is unlimited, allowing the user to reflect the logical
structure of the actual system in the model structure.

The model structure is described in {\opp}'s NED language.

Modules that contain submodules are called \textit{compound
  modules}\index{module!compound}, as opposed to \textit{simple
  modules}\index{module!simple} at the lowest level of the
module hierarchy. Simple modules contain the algorithms of the model.
The user implements the simple modules in C++, using the {\opp}
simulation class library.


\subsection{Module Types}
\label{sec:overview:module-types}
\index{module!types}

Both simple and compound modules are instances of \textit{module
  types}. In describing the model, the user defines module types;
instances of these module types serve as components for more complex
module types. Finally, the user creates the system module as an
instance of a previously defined module type; all modules in the
network are instantiated as submodules and sub-submodules of the
system module.

When a module type is used as a building block, it makes no difference
whether it is a simple or compound module. This allows
the user to split a simple module into several
simple modules embedded in a compound\index{module!compound} module,
or vice versa, to aggregate the functionality of a compound module into a
single simple module, without affecting existing users of the module
type.

Module types can be stored in files separate from the location
of their actual usage. This means that the user can group existing
module types and create \textit{component libraries}\index{module!libraries}.
This feature will be discussed later, in chapter \ref{cha:run-sim}.



\subsection{Messages, Gates, Links}
\label{sec:overview:messages-gates-links}

Modules communicate by exchanging
\textit{messages}\index{message!exchanging}. In an actual simulation,
messages can represent frames or packets in a computer network, jobs
or customers in a queuing network, or other types of mobile entities.
Messages can contain arbitrarily complex data structures. Simple
modules can send messages either directly to their destination or
along a predefined path, through gates and connections.


The ``local simulation time'' of a module advances when the module
receives a message. The message can arrive from another module
or from the same module (\textit{self-messages} are used to implement
timers).


\textit{Gates}\index{gate} are the input and output interfaces of
modules; messages are sent out through output gates and arrive through
input gates.

Each \textit{connection}\index{connection} (also called
\textit{link}\index{link}) is created within a single level of the
module hierarchy: within a compound module, you can connect the
corresponding gates of two submodules, or a gate of one submodule and
a gate of the compound module (Fig.
\ref{fig:ch-overview:modules}).

Because of the hierarchical structure of the model, messages typically
travel through a series of connections, starting and arriving in simple
modules. Compound modules act like ``cardboard boxes'' in the model,
transparently relaying messages between their inner realm and the
outside world.


\subsection{Modeling of Packet Transmissions}
\label{sec:overview:modeling-of-packet-transmissions}

To facilitate the modeling of communication networks, connections
can be used to model physical links. Connections support
the following parameters: \textit{data rate}, \textit{propagation delay},
\textit{bit error rate}, and \textit{packet error rate}, and may be
disabled. These parameters and the underlying algorithms are encapsulated
into \textit{channel} objects. The user can parameterize the channel
types provided by {\opp}, and also create new ones.

When data rates are used, a packet object is by default delivered to the
target module at the simulation time that corresponds to the end of the
packet reception. Since this behavior is not suitable for the modeling of
some protocols (e.g. half-duplex Ethernet), {\opp} provides the
possibility for the target module to specify that it wants the packet object
to be delivered to it when the packet reception starts.


\subsection{Parameters}
\label{sec:overview:parameters}
\index{module!parameters}
\index{parameters|see{module parameters}}

Modules can have parameters. Parameters can be assigned in either
the NED files or the configuration file \ttt{omnetpp.ini}.

Parameters can be used to customize simple module behavior
and to parameterize the model's topology.

Parameters can hold string, numeric, or boolean
values or can
contain XML data trees. Numeric values include expressions using
other parameters and calling C functions, random variables from
different distributions, and values input interactively by the user.

Numeric-valued parameters can be used to construct topologies in a
flexible way. Within a compound module, parameters can define the
number of submodules, number of gates, and the way the internal
connections are made.


\subsection{Topology Description Method}
\label{sec:overview:topology-description-method}
\index{topology!description}

The user defines the structure of the model in NED language descriptions
(Network Description). The NED language will be discussed in detail
in chapter \ref{cha:ned-lang}.


\section{Programming the Algorithms}
\label{sec:overview:programmable-using-cplusplus}

The simple\index{module!simple} modules of a model contain algorithms
implemented as C++ functions.
The full flexibility and power of the programming language can
be utilized, supported by the {\opp} simulation class library.
The simulation programmer can choose between event-driven and process-style
descriptions and freely use object-oriented concepts
(inheritance, polymorphism, etc.) and design patterns to extend the
functionality of the simulator.

Simulation objects (messages, modules, queues, etc.) are represented
by C++ classes. They have been designed to work together efficiently,
creating a powerful simulation programming framework.
The following classes are part of the simulation class library:

\begin{itemize}
  \item module, gate, parameter, channel
  \item message, packet
  \item container classes (e.g. queue, array)
  \item data collection classes
  \item statistic and distribution estimation classes (histograms, $P^2$
        algorithm for calculating quantiles, etc.)
\end{itemize}

The classes are also specially instrumented, allowing one
to traverse objects of a running simulation and display information
about them such as name, class name, state variables, or contents.
This feature makes it possible to create a simulation GUI where
all internals of the simulation are visible.


% \subsection{Creating Simple Modules}
% \label{sec:overview:creating-simple-modules}
% \index{module!simple!creation}
%
% Each simple\index{module!simple} module type is implemented with a C++ class. Simple
% module classes are derived from a simple module base class, by
% redefining the virtual function that contains the algorithm.
% The user can add other member functions to the class to split
% up a complex algorithm; he can also add data members to the class.
%
% It is also possible to derive new simple\index{module!simple} module classes from
% existing ones. For example, if one wants to experiment with retransmission
% timeout schemes in a transport protocol, he can implement the
% protocol in one class, create a virtual function for the retransmission
% algorithm, and then derive a family of classes that implement
% concrete schemes. This concept is further supported by the fact
% that in the network description, actual module types can be parameters.
%
%
% \subsection{Object Mechanisms}
% \label{sec:overview:object-mechanisms}
%
% The use of smart container classes allows the user to build
% \textit{aggregate data structures}\index{aggregate data structures}.
% For example, one can add any number of objects to a message object as
% parameters. Since the added objects can contain further objects,
% complex data structures can be built.
%
% There is an efficient \textit{ownership}\index{ownership} mechanism
% built in. The user can specify an owner for each object; then, the
% owner object will have the responsibility of destroying that object.
% Most of the time, the ownership mechanism works transparently;
% ownership only needs to be explicitly managed when the user wants to
% do something non-typical.
%
%
% The \textit{foreach}\index{forEachChild mechanism} mechanism allows one to
% enumerate the objects inside a container object in a uniform way and
% perform operations on them. This feature makes it possible to
% handle many objects together. (The \textit{foreach} feature is extensively used
% by the user interfaces with debugging capability and the snapshot
% mechanism; see later.)
%
%
% \subsection{Deriving New Classes}
% \label{sec:overview:derive-new-classes}
%
% In most cases, the functionality offered by the {\opp} classes
% is sufficient for the user. But if needed, one can derive new
% classes from the existing ones or create entirely new classes.
% For flexibility, several member functions are declared virtual.
% When the user creates new classes, certain rules need to be followed
% so that the object can work with other objects seamlessly.
%
%
% \subsection{Self-describing Objects to Facilitate Debugging}
% \label{sec:overview:self-describing-objects-to-ease-debugging}
% \index{debugging}
%
%
%
% A unique feature called \textit{snapshot}\index{snapshot} allows the
% user to dump the contents of the simulation model or a part of it into
% a text file. The file will contain textual reports about every object;
% this can be of invaluable help during debugging. Ordinary
% variables can also be made to appear in the snapshot file. Snapshot
% creations can be scheduled from within the simulation program or done
% from the user interface.
%


\section{Using {\opp}}
\label{sec:overview:using-omnetpp}


\subsection{Building and Running Simulations}
\label{sec:overview:building-and-running-simulations}
\index{simulation!building}
\index{simulation!running}

This section provides insights into working with {\opp} in practice.
Issues such as model files and compiling and running simulations are
discussed.

An {\opp} model consists of the following parts:
\begin{itemize}
  \item{NED language topology description(s)\index{ned!files} (\texttt{.ned} files)
    that describe the module structure with parameters, gates, etc.
    NED files can be written using any text editor, but the {\opp} IDE
    provides excellent support for two-way graphical and text editing.}
  \item{Message definitions (\texttt{.msg} files) that let one define message
    types and add data fields to them. {\opp} will translate message definitions
    into full-fledged C++ classes.}
  \item{Simple module sources. They are C++ files, with \texttt{.h}/\texttt{.cc} suffix.}
\end{itemize}

The simulation system provides the following components:
\begin{itemize}
  \item{Simulation kernel\index{simulation!kernel}. This contains the
    code that manages the simulation and the simulation class library.
    It is written in C++, compiled into a shared or static library.}
  \item{User interfaces\index{simulation!user interface}.
    \index{user interface} {\opp} user interfaces
    are used in simulation execution, to facilitate debugging,
    demonstration, or batch execution of simulations. They are
    written in C++, compiled into libraries.}
\end{itemize}


Simulation programs are built from the above components. First,
\ttt{.msg} files are translated into C++ code using the \ttt{opp\_msgc}.
program. Then all C++ sources are compiled and linked with the simulation
kernel and a user interface library to form a simulation executable or
shared library. NED files\index{ned!files} are loaded dynamically in their original
text forms when the simulation program starts.


\subsubsection{Running the Simulation and Analyzing the Results}
\label{sec:overview:running-simulation-and-analyzing-results}

The simulation may be compiled as a standalone program executable,
or as a shared library to be run using {\opp}'s \fprog{opp\_run} utility.
When the program is started, it first reads the NED files\index{ned!files},
then the configuration file\index{simulation!configuration file} usually called
\ffilename{omnetpp.ini}. The configuration file contains settings that
control how the simulation is executed, values for model parameters, etc.
The configuration file can also prescribe several simulation runs; in
the simplest case, they will be executed by the simulation program one
after another.

The output of the simulation is written into result files: output vector
files\index{output!vector file}, output scalar files\index{output!scalar file},
and possibly the user's own output files.
{\opp} contains an Integrated Development Environment (IDE) that provides
a rich environment for analyzing these files. Output files are line-oriented
text files which makes it possible to process them with a variety of tools
and programming languages as well, including Matlab, GNU R, Perl, Python,
and spreadsheet programs.


\subsubsection{User Interfaces}
\label{sec:overview:user-interfaces}
\index{simulation!user interface}

The primary purpose of user interfaces is to make the internals
of the model visible to the user, to control the simulation execution,
and possibly allow the user to intervene by changing variables/objects
inside the model. This is very important in the development/debugging
phase of the simulation project. Equally important, a hands-on
experience allows the user to get a feel of the model's
behavior. The graphical user interface can also be used to
demonstrate a model's operation.


The same simulation model can be executed with various user
interfaces, with no change in the model files themselves.
The user would typically test and debug the simulation with a powerful
graphical user interface, and finally run it with a simple,
fast user interface that supports batch execution.


\subsubsection{Component Libraries}
\label{sec:overview:component-libraries}
\index{module!libraries}

Module types can be stored in files separate from the place
of their actual use, enabling the user to group existing
module types and create component libraries.


\subsubsection{Universal Standalone Simulation Programs}
\label{sec:overview:universal-standalone-simulation-programs}


A simulation executable can store several independent models
that use the same set of simple modules. The user can specify
in the configuration file which model is to be run. This
allows one to build one large executable that contains several
simulation models, and distribute it as a standalone simulation
tool. The flexibility of the topology description language also
supports this approach.


\subsection{What Is in the Distribution}
\label{sec:overview:what-is-in-distribution}

An {\opp} installation contains the following subdirectories. Depending
on the platform, there may also be additional directories present, containing
software bundled with {\opp}.)

The simulation system itself:

\begin{Verbatim}[commandchars=\\\{\}]
  \tbf{omnetpp/}         {\opp} root directory
    \tbf{bin/}           {\opp} executables
    \tbf{include/}       header files for simulation models
    \tbf{lib/}           library files
    \tbf{images/}        icons and backgrounds for network graphics
    \tbf{doc/}           manuals, readme files, license, APIs, etc.
      \tbf{ide-customization-guide/} how to write new wizards for the IDE
      \tbf{ide-developersguide/} writing extensions for the IDE
      \tbf{manual/}      manual in HTML
      \tbf{ned2/}        DTD definition of the XML syntax for NED files
      \tbf{tictoc-tutorial/}  introduction to using {\opp}
      \tbf{api/}         API reference in HTML
      \tbf{nedxml-api/}  API reference for the NEDXML library
      \tbf{parsim-api/}  API reference for the parallel simulation library
    \tbf{src/}           {\opp} sources
      \tbf{sim/}         simulation kernel
        \tbf{parsim/}    files for distributed execution
        \tbf{netbuilder/}files for dynamically reading NED files
      \tbf{envir/}       common code for user interfaces
      \tbf{cmdenv/}      command-line user interface
      \tbf{qtenv/}       Qt-based user interface
      \tbf{nedxml/}      NEDXML library, opp_nedtool, opp_msgtool
      \tbf{scave/}       result analysis library, opp_scavetool
      \tbf{eventlog/}    eventlog processing library
      \tbf{layout/}      graph layouter for network graphics
      \tbf{common/}      common library
      \tbf{utils/}       opp_makemake, opp_test, etc.
    \tbf{ide/}           Simulation IDE
    \tbf{python/}        Python libraries for {\opp}
      \tbf{omnetpp/}     Python package name
        \tbf{scave/}     Python API for result analysis
	...
    \tbf{test/}          Regression test suite
      \tbf{core/}        tests for the simulation library
      \tbf{anim/}        tests for graphics and animation
      \tbf{dist/}        tests for the built-in distributions
      \tbf{makemake/}    tests for opp_makemake
      ...
\end{Verbatim}

The Eclipse-based Simulation IDE is in the \texttt{ide} directory.

\begin{Verbatim}[commandchars=\\\{\}]
    \tbf{ide/}           Simulation IDE
      \tbf{features/}    Eclipse feature definitions
      \tbf{plugins/}     IDE plugins (extensions to the IDE can be dropped here)
      ...
\end{Verbatim}

The Windows version of {\opp} contains a redistribution of the MinGW
gcc compiler, together with a copy of MSYS that provides Unix tools
commonly used in Makefiles. The MSYS directory also contains various
3rd party open-source libraries needed to compile and run {\opp}.

\begin{Verbatim}[commandchars=\\\{\}]
    \tbf{tools/}       Platform-specific tools and compilers (e.g. MinGW/MSYS on Windows)
\end{Verbatim}

Sample simulations are in the \texttt{samples} directory.

\begin{Verbatim}[commandchars=\\\{\}]
    \tbf{samples/}     directories for sample simulations
      \tbf{aloha/}     models the Aloha protocol
      \tbf{cqn/}       Closed Queueing Network
      ...
\end{Verbatim}

The \texttt{contrib} directory contains material from the {\opp} community.

\begin{Verbatim}[commandchars=\\\{\}]
    \tbf{contrib/}     directory for contributed material
      \tbf{akaroa/}    Patch to compile akaroa on newer gcc systems
      \tbf{topologyexport/}  Export the topology of a model in runtime
      ...
\end{Verbatim}



%%% Local Variables:
%%% mode: latex
%%% TeX-master: "usman"
%%% End:
